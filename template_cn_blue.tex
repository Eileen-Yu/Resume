%# -*- coding:utf-8 -*-
%% start of file `template_en.tex'.
%% Copyright 2006-1008 Xavier Danaux (xdanaux@gmail.com).
%
% This work may be distributed and/or modified under the
% conditions of the LaTeX Project Public License version 1.3c,
% available at http://www.latex-project.org/lppl/.


\documentclass[11pt,a4paper]{moderncv}

\usepackage{fontspec,xunicode}
\setmainfont{Tahoma}
\usepackage[slantfont,boldfont]{xeCJK}
\usepackage{xcolor}                 % replace by the encoding you are using


\setmainfont{Times New Roman}%缺省英文字体.serif是有衬线字体sans serif无衬线字体
\setCJKmainfont[ItalicFont={Kai}, BoldFont={Hei}]{STSong}%衬线字体 缺省中文字体为
\setCJKsansfont{STSong}
\setCJKmonofont{STFangsong}%中文等宽字体
%-----------------------xeCJK下设置中文字体------------------------------%
\setCJKfamilyfont{song}{SimSun}                             %宋体 song
\newcommand{\song}{\CJKfamily{song}}
\setCJKfamilyfont{fs}{FangSong_GB2312}                      %仿宋2312 fs
\newcommand{\fs}{\CJKfamily{fs}}
\setCJKfamilyfont{yh}{Microsoft YaHei}                    %微软雅黑 yh
\newcommand{\yh}{\CJKfamily{yh}}
\setCJKfamilyfont{hei}{SimHei}                              %黑体  hei
\newcommand{\hei}{\CJKfamily{hei}}
\setCJKfamilyfont{hwxh}{STXihei}                                %华文细黑  hwxh
\newcommand{\hwxh}{\CJKfamily{hwxh}}
\setCJKfamilyfont{asong}{Adobe Song Std}                        %Adobe 宋体  asong
\newcommand{\asong}{\CJKfamily{asong}}
\setCJKfamilyfont{ahei}{Adobe Heiti Std}                            %Adobe 黑体  ahei
\newcommand{\ahei}{\CJKfamily{ahei}}
\setCJKfamilyfont{akai}{Adobe Kaiti Std}                            %Adobe 楷体  akai
\newcommand{\akai}{\CJKfamily{akai}}


%------------------------------设置字体大小------------------------%
\newcommand{\chuhao}{\fontsize{42pt}{\baselineskip}\selectfont}     %初号
\newcommand{\xiaochuhao}{\fontsize{36pt}{\baselineskip}\selectfont} %小初号
\newcommand{\yihao}{\fontsize{28pt}{\baselineskip}\selectfont}      %一号
\newcommand{\erhao}{\fontsize{21pt}{\baselineskip}\selectfont}      %二号
\newcommand{\xiaoerhao}{\fontsize{18pt}{\baselineskip}\selectfont}  %小二号
\newcommand{\sanhao}{\fontsize{15.75pt}{\baselineskip}\selectfont}  %三号
\newcommand{\sihao}{\fontsize{14pt}{\baselineskip}\selectfont}         %四号
\newcommand{\xiaosihao}{\fontsize{12pt}{\baselineskip}\selectfont}  %小四号
\newcommand{\wuhao}{\fontsize{10.5pt}{\baselineskip}\selectfont}    %五号
\newcommand{\subwuhao}{\fontsize{10pt}{\baselineskip}\selectfont}    %次五号
\newcommand{\xiaowuhao}{\fontsize{9pt}{\baselineskip}\selectfont}   %小五号
\newcommand{\liuhao}{\fontsize{7.875pt}{\baselineskip}\selectfont}  %六号
\newcommand{\qihao}{\fontsize{5.25pt}{\baselineskip}\selectfont}    %七号


%\usepackage{fontawesome}
% \setCJKmainfont[BoldFont={WenQuanYi Micro Hei/Bold}]{WenQuanYi Micro Hei}
%\defaultfontfeatures{Mapping=tex-text}
%\XeTeXlinebreaklocale "zh"
%\XeTeXlinebreakskip = 0pt plus 1pt minus 0.1pt
% moderncv themes
\moderncvtheme[blue]{classic}                 % optional argument are 'blue' (default), 'orange', 'red', 'green', 'grey' and 'roman' (for roman fonts, instead of sans serif fonts)
%\moderncvtheme[green]{classic}                % idem
%\moderncvtheme[blue,roman]{hht}
% character encoding



% adjust the page margins
\usepackage[scale=0.9]{geometry}
%\setlength{\hintscolumnwidth}{3cm}						% if you want to change the width of the column with the dates
%\AtBeginDocument{\setlength{\maketitlenamewidth}{6cm}}  % only for the classic theme, if you want to change the width of your name placeholder (to leave more space for your address details
\AtBeginDocument{\recomputelengths}                     % required when changes are made to page layout lengths

% personal data
\firstname{于}
\familyname{力钧}
\title{Eileen Yu}               % optional, remove the line if not wanted
% \address{杭州}{}    % optional, remove the line if not wanted
% \address{1990/11/11}{}    % optional, remove the line if not wanted
\mobile{13795281028}                    % optional, remove the line if not wanted
%\fax{fax (optional)}                          % optional, remove the line if not wanted
\email{ylj3331@sjtu.edu.cn}                     % optional, remove the line if not wanted
% \homepage{Blog: http://geekplux.com} % optional, remove the line if not wanted
% \social[github]{GitHub: https://github.com/geekplux}
%\extrainfo{%
  %LinkedIn: https://cn.linkedin.com/in/xxx \\
  %WeChat: xxxx \\
  %QQ: 123456
%}

%\photo[100pt]{avatar.png}                         % '64pt' is the height the picture must be resized to and 'picture' is the name of the picture file; optional, remove the line if not wanted
%\quote{China\TeX 您的LaTeX乐园,TeX\&\LaTeX 王国}                 % optional, remove the line if not wante

%\nopagenumbers{}                             % uncomment to suppress automatic page numbering for CVs longer than one page


%----------------------------------------------------------------------------------
%            content
%----------------------------------------------------------------------------------
\begin{document}
\maketitle
\vspace*{-14mm}

\section{教育经历}
\cventry{17.09-\textbf{至今}}{本科}{上海交通大学}{媒体与传播学院}{传播学}{}                % arguments 3 to 6 are optional
%\cvlistitem{最快编程大师一等奖}
%\cvlistitem{最强编程大师金奖}
%\cvlistitem{第 x 届「编程杯」gayhub 赛区一等奖}
%\cvlistitem{国家奖学金/三好学生/学生会主席/\emph{获得女朋友一个}}
%\cventry{15.09-18.06}{硕士}{和尚庙大学}{软件工程}{实验室 XXX 导师 XXX}{主要研究了
% 人工智能,图形学,编译原理,机械键盘的拆装,快递包装的暴力拆解,颈椎与视觉保
% 养,抹平小腹,治疗腰椎间盘突出}                % arguments 3 to 6 are optional


\section{技能}
\cvline{\textbf{语言}}{CET-4 (650+) | CET-6 (600+)}
\cvline{\textbf{软件}}{熟练使用SPSS | 精通MS Office | 熟练运用PS | 熟练运用Premiere}
\cvline{\textbf{其他}}{熟练掌握摄影技能 | 良好的文字功底}

\section{项目经历}
\cventry{17.09-18.01}{鲜花电商市场营销状况调研}{}{}{}{\hspace{0.7cm}对2017年四季度的电商鲜花市场营销状况进行全面分析及预测趋势。前期通过媒介监测获取数据监测报告,使用焦点小组访谈形式收集相关信息;中期对线上线下融合的鲜花电商进行实地调研、搜集反馈信息、收集案例;后期综合数据信息、绘制趋势图,撰写总结报告并提出市场预测。}
\cventry{18.06-18.12}{社交媒介、媒介效果与认知研究}{}{}{}{\hspace{0.7cm}探讨媒介用户行为背后的人格特质与认知心理。通过WOS、EBSCO、Taylor\&Francis等数据库进行核心期刊论文检索,对媒介渠道特征、人格特质、用户行为进行归类整理,在此基础上将心理学与传播学相结合,立足于“使用与满足”理论,从自我认同、社会认同、个人调节、社会融入四个维度设计用户行为动机Likert五级量表,同时引入大五人格量表,回收300份有效样本且使用SPSS进行Pearson多维度相关分析。}
\cventry{18.09-18.12}{上海交通大学院级部门组织诊断}{}{}{}{\hspace{0.7cm}独立完成学院组织诊断,发现组织架构、绩效、文化、人际关系等方面的问题并提出相关建议。前期通过在线问卷对组织内部人员进行相关调查,确立七个相关维度,并进行访谈设计。对员工、管理层在内的40位人员进行面对面访谈,对相关问题进行频次统计并整理记录,针对性提出建议,撰写诊断报告。}
\cventry{18.10-18.12}{光明品牌公关调研与策划}{}{}{}{\hspace{0.7cm}对光明乳业公关状况进行全面调研并使用ROPE模式提出一个完整的公关活动策划案。通过媒介监测、报表分析、数据收集基本了解企业的公关运营情况,发放问卷获得200份有效样本,利用SPSS进行ANOVA分析与多元线性回归分析,确立公关目标群体,制定公关主题、宣传语及线上线下联动合作活动执行方案,提出公关效果预期与评估。}
\cventry{19.02-19.04}{针对落后地区儿童的短视频栏目策划}{}{}{}{\hspace{0.7cm}以农村儿童为核心受众,策划公益教育方向的短视频栏目及完成拍摄,并进行效果检验。前期实地观察调研,查阅相关文献与现有视频,完成栏目选题设计、内容创作,结合扩散途径理论选择合适的新媒体传播方式,并对儿童进行检验,以研究数字鸿沟的消除状况。}


\section{实践}
\subsection{校内实践}
\cventry{18.12-\textbf{至今}}{SJTU中银俱乐部}{文宣部部长}{}{}{负责俱乐部的自媒体运营;带领团队完成选题、拍摄、撰稿、推送编辑;负责拍摄社团影视作品。}
\cventry{17.12-19.04}{上海交通大学学生联合会跨文化交流中心}{秘书部部长}{}{}{活动专题策划;组织内建策划;对外联系合作}
\subsection{社会实践}
\cvline{17.11}{\textbf{上海交大-国际传播学会(ICA)新媒体国际论坛}——嘉宾接待}
\cvline{18.07}{\textbf{上海交大-国际传播学会(ICA)新媒体国际论坛}——论文集编辑校对工作}
\cvline{18.05}{\textbf{上海科技节-科普大讲坛}——主持人}
\cvline{12\textbf{年}-\textbf{至今}}{\textbf{微信个人公众号运营}}




%\cvline{markvis}{在 markdown 中直接生成可视化图表的插件
%  \emph{https://markvis.js.org} \textbf{GitHub 1000 stars}}
%\cvline{netjsongraph.js}{用力导向图可视化出无线路由图谱数据 \emph{https://github.com/netjson/netjsongraph.js}}
%\cvline{typing}{Hexo 静态博客主题 \emph{https://github.com/geekplux/hexo-theme-typing}}
%\cvline{UnityVis}{Unity 中的基本可视化图表 \emph{https://github.com/geekplux/Basic-Visualization-in-Unity}}

%\section{Publications}
%\cvline{已录用}{张三,李四,王麻子 基于 latex 的简历凑字数研究[C]// CVChina. 2017.}

% \subsection{Vocational}
% \cventry{year--year}{Job title}{Employer}{City}{}{Description}                % arguments 3 to 6 are optional
% \cventry{year--year}{Job title}{Employer}{City}{}{Description}                % arguments 3 to 6 are optional
% \subsection{Miscellaneous}
% \cventry{year--year}{Job title}{Employer}{City}{}{Description line 1\newline{}Description line 2}% arguments 3 to 6 are optional

% \section{Languages}
% \cvlanguage{language 1}{Skill level}{Comment}
% \cvlanguage{language 2}{Skill level}{Comment}
% \cvlanguage{language 3}{Skill level}{Comment}

% \section{Computer skills}
% \cvcomputer{category 1}{XXX, YYY, ZZZ}{category 4}{XXX, YYY, ZZZ}
% \cvcomputer{category 2}{XXX, YYY, ZZZ}{category 5}{XXX, YYY, ZZZ}
% \cvcomputer{category 3}{XXX, YYY, ZZZ}{category 6}{XXX, YYY, ZZZ}

% \section{Interests}
% \cvline{篮球}{\small 体力与技巧}
% \cvline{hobby 2}{\small Description}
% \cvline{hobby 3}{\small Description}

% \renewcommand{\listitemsymbol}{-} % change the symbol for lists

% \section{Extra 1}
% \cvlistitem{Item 1}
% \cvlistitem{Item 2}
%\cvlistitem[+]{Item 3}            % optional other symbol% XeLaTeX can use any Mac OS X font. See the setromanfont command below.
% Input to XeLaTeX is full Unicode, so Unicode characters can be typed directly into the source.

% The next lines tell TeXShop to typeset with xelatex, and to open and save the source with Unicode encoding.

%!TEX TS-program = xelatex
%!TEX encoding = UTF-8 Unicode

%\section{Extra 2}
%\cvlistdoubleitem[\Neutral]{Item 1}{Item 4}
%\cvlistdoubleitem[\Neutral]{Item 2}{Item 5}
%\cvlistdoubleitem[\Neutral]{Item 3}{}

%% Publications from a BibTeX file
%\nocite{*}
%\bibliographystyle{plain}
%\bibliography{publications}       % 'publications' is the name of a BibTeX file

% \begin{thebibliography}{}
% \bibitem[]{} 移动增强现实可视化综述[C]. ChinaVis 2017.
% \end{thebibliography}


\end{document}


%% end of file `template_en.tex'.

%%% Local Variables:
%%% mode: latex
%%% TeX-command-extra-options: "-shell-escape"
%%% TeX-master: t
%%% TeX-engine: xetex
%%% End:
