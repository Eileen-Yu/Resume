%# -*- coding:utf-8 -*-
%% start of file `template_en.tex'.
%% Copyright 2006-1008 Xavier Danaux (xdanaux@gmail.com).
%
% This work may be distributed and/or modified under the
% conditions of the LaTeX Project Public License version 1.3c,
% available at http://www.latex-project.org/lppl/.


\documentclass[11pt,a4paper]{moderncv}

\usepackage{fontspec,xunicode}
\setmainfont{Tahoma}
\usepackage[slantfont,boldfont]{xeCJK}
\usepackage{xcolor}                 % replace by the encoding you are using


\setmainfont{Times New Roman}%缺省英文字体.serif是有衬线字体sans serif无衬线字体
\setCJKmainfont[ItalicFont={Kai}, BoldFont={Hei}]{STSong}%衬线字体 缺省中文字体为
\setCJKsansfont{STSong}
\setCJKmonofont{STFangsong}%中文等宽字体
%-----------------------xeCJK下设置中文字体------------------------------%
\setCJKfamilyfont{song}{SimSun}                             %宋体 song
\newcommand{\song}{\CJKfamily{song}}
\setCJKfamilyfont{fs}{FangSong_GB2312}                      %仿宋2312 fs
\newcommand{\fs}{\CJKfamily{fs}}
\setCJKfamilyfont{yh}{Microsoft YaHei}                    %微软雅黑 yh
\newcommand{\yh}{\CJKfamily{yh}}
\setCJKfamilyfont{hei}{SimHei}                              %黑体  hei
\newcommand{\hei}{\CJKfamily{hei}}
\setCJKfamilyfont{hwxh}{STXihei}                                %华文细黑  hwxh
\newcommand{\hwxh}{\CJKfamily{hwxh}}
\setCJKfamilyfont{asong}{Adobe Song Std}                        %Adobe 宋体  asong
\newcommand{\asong}{\CJKfamily{asong}}
\setCJKfamilyfont{ahei}{Adobe Heiti Std}                            %Adobe 黑体  ahei
\newcommand{\ahei}{\CJKfamily{ahei}}
\setCJKfamilyfont{akai}{Adobe Kaiti Std}                            %Adobe 楷体  akai
\newcommand{\akai}{\CJKfamily{akai}}


%------------------------------设置字体大小------------------------%
\newcommand{\chuhao}{\fontsize{42pt}{\baselineskip}\selectfont}     %初号
\newcommand{\xiaochuhao}{\fontsize{36pt}{\baselineskip}\selectfont} %小初号
\newcommand{\yihao}{\fontsize{28pt}{\baselineskip}\selectfont}      %一号
\newcommand{\erhao}{\fontsize{21pt}{\baselineskip}\selectfont}      %二号
\newcommand{\xiaoerhao}{\fontsize{18pt}{\baselineskip}\selectfont}  %小二号
\newcommand{\sanhao}{\fontsize{15.75pt}{\baselineskip}\selectfont}  %三号
\newcommand{\sihao}{\fontsize{14pt}{\baselineskip}\selectfont}         %四号
\newcommand{\xiaosihao}{\fontsize{12pt}{\baselineskip}\selectfont}  %小四号
\newcommand{\wuhao}{\fontsize{10.5pt}{\baselineskip}\selectfont}    %五号
\newcommand{\subwuhao}{\fontsize{10pt}{\baselineskip}\selectfont}    %次五号
\newcommand{\xiaowuhao}{\fontsize{9pt}{\baselineskip}\selectfont}   %小五号
\newcommand{\liuhao}{\fontsize{7.875pt}{\baselineskip}\selectfont}  %六号
\newcommand{\qihao}{\fontsize{5.25pt}{\baselineskip}\selectfont}    %七号


%\usepackage{fontawesome}
% \setCJKmainfont[BoldFont={WenQuanYi Micro Hei/Bold}]{WenQuanYi Micro Hei}
%\defaultfontfeatures{Mapping=tex-text}
%\XeTeXlinebreaklocale "zh"
%\XeTeXlinebreakskip = 0pt plus 1pt minus 0.1pt
% moderncv themes
\moderncvtheme[blue]{classic}                 % optional argument are 'blue' (default), 'orange', 'red', 'green', 'grey' and 'roman' (for roman fonts, instead of sans serif fonts)
%\moderncvtheme[green]{classic}                % idem
%\moderncvtheme[blue,roman]{hht}
% character encoding



% adjust the page margins
\usepackage[scale=0.9]{geometry}
%\setlength{\hintscolumnwidth}{3cm}						% if you want to change the width of the column with the dates
%\AtBeginDocument{\setlength{\maketitlenamewidth}{6cm}}  % only for the classic theme, if you want to change the width of your name placeholder (to leave more space for your address details
\AtBeginDocument{\recomputelengths}                     % required when changes are made to page layout lengths

% personal data
\firstname{于}
\familyname{力钧}
\title{Eileen Yu}               % optional, remove the line if not wanted
% \address{杭州}{}    % optional, remove the line if not wanted
% \address{1990/11/11}{}    % optional, remove the line if not wanted
\mobile{13795281028}                    % optional, remove the line if not wanted
%\fax{fax (optional)}                          % optional, remove the line if not wanted
\email{ylj3331@sjtu.edu.cn}                     % optional, remove the line if not wanted
% \homepage{Blog: http://geekplux.com} % optional, remove the line if not wanted
% \social[github]{GitHub: https://github.com/geekplux}
%\extrainfo{%
  %LinkedIn: https://cn.linkedin.com/in/xxx \\
  %WeChat: xxxx \\
  %QQ: 123456
%}

%\photo[100pt]{avatar.png}                         % '64pt' is the height the picture must be resized to and 'picture' is the name of the picture file; optional, remove the line if not wanted
%\quote{China\TeX 您的LaTeX乐园,TeX\&\LaTeX 王国}                 % optional, remove the line if not wante

%\nopagenumbers{}                             % uncomment to suppress automatic page numbering for CVs longer than one page


%----------------------------------------------------------------------------------
%            content
%----------------------------------------------------------------------------------
\begin{document}
\maketitle
\vspace*{-14mm}

\section{教育经历}
\cventry{17.09-\textbf{至今}}{本科}{上海交通大学}{媒体与传播学院}{传播学}{GPA 90/100}                % arguments 3 to 6 are optional
%\cvlistitem{最快编程大师一等奖}
%\cvlistitem{最强编程大师金奖}
%\cvlistitem{第 x 届「编程杯」gayhub 赛区一等奖}
%\cvlistitem{国家奖学金/三好学生/学生会主席/\emph{获得女朋友一个}}
%\cventry{15.09-18.06}{硕士}{和尚庙大学}{软件工程}{实验室 XXX 导师 XXX}{主要研究了
% 人工智能,图形学,编译原理,机械键盘的拆装,快递包装的暴力拆解,颈椎与视觉保
% 养,抹平小腹,治疗腰椎间盘突出}                % arguments 3 to 6 are optional


\section{技能}
\cvline{\textbf{语言}}{ 中文(母语) | 英语(CET-6 600+)}
\cvline{\textbf{软件}}{Python | SPSS | MS Office | PS | Premiere}
\cvline{\textbf{其他}}{熟练掌握摄影技能 | 良好的文字功底}

\section{实习经历}
\cventry{19.06-\textbf{至今}}{QuestMobile}{数据分析师}{}{}{\begin{itemize}
   \item[•] 全程跟进B站战略调整咨询项目,采集整理5000余条数据,开展2组焦点小组访谈,设计并发放回收1500余份问卷,输出长达逾50页的咨询建议报告。
   \item[-]深入研究10多份投资报告,用PEST模型完成市场调研和收费模式分析以预测盈亏平衡点;对不同领域的典型企业进行运营模式与价值链分析,筛选出8个主要竞品,并通过公司系统和公开报告,分析其过去3-5年内的执行项目、市场表现和包括损益表、资产负债表、现金流量表等在内的财务指标。
   \item[-]从供给端与需求端内容、媒介渠道、品牌定位和竞品四个维度设计研究框架,并全程跟进项目,撰写输出报告。
\end{itemize}}


\section{项目经历}

\cventry{17.09-18.01}{鲜花电商市场营销状况调研}{}{}{}{\begin{itemize}
   \item[•] 带领小组完成对2017年第四季度鲜花电商市场营销状况的整体性分析,覆盖包含政策、增长趋势、用户画像、产品与服务等分析维度,探索可能存在的利基市场。
   \item[-] 爬取并整合主流鲜花电商的1000余条数据,立足于4C和4P理论对各维度进行数据比对,深入挖掘用户需求。
   \item[-] 基于案例研究和田野调查,设计焦点小组访谈,深入探究鲜花电商客户满意度。
   \item[-] 从两种不同的商业模式出发设计两套发展方案,评估各方案的可行性并输出20余页的报告。
\end{itemize}}


\cventry{19.06-\textbf{至今}}{短视频与社交营销大数据分析}{SJTU大数据与传播创新实验室(T-lab)}{}{}{\begin{itemize}
   \item[•] 使用Python及相关统计工具,基于使用与满足理论与创新扩散理论,分析以抖音为代表的短视频APP的社交营销效果。
   \item[-] 收集整合3000余条数据,从性别、年龄、地域、使用频率与时长、使用时段等维度对用户进行画像及标签归类。
   \item[-] 爬取意见领袖私域流量构成和带货转化率,爬取十余类商品销售趋势并划分量级,与用户属性交叉分析。
\end{itemize}}

%\cventry{18.10-18.12}{光明品牌公关调研与策划}{}{}{}{\begin{itemize}
%   \item[•] 以小组形式通过现有销售数据和公开报告,对光明品牌公关现状进行深入解析,基于ROPE模型提出一份完整公关活动份策划案。
%   \item[-] 通过媒介监测与桌面调研,获得企业与客户、媒体公共关系的基本运营情况,确立活动目标群体。
%   \item[-] 设计问卷深入调研客户人口统计特征、购买行为与触媒习惯,回收200余份有效问卷,利用SPSS进行ANOVA与多元线性回归分析。
%   \item[-] 设计线上线下联动活动的主题、标语与具体方案内容,提出公关效果预期与评估。
%\end{itemize}}

\cventry{18.10-18.12}{针对落后地区儿童的短视频教育策划}{}{}{}{\begin{itemize}
   \item[•] 一项针对落后地区儿童公共教育问题的短视频策划方案,包括视频拍摄和效果监测。
   \item[-] 通过田野观察、访问调查、查阅相关文献和视频内容获得偏远地区儿童的教育背景和现状。
   \item[-] 立足于相关传播理论与一系列预验,挑选语言运用作为合适主题,选择在线课程平台作为新媒体传播渠道。
   \item[-] 导演并拍摄有关成语故事的短视频,设计在线课后练习,检验课程项目的传播效果,以研究数字鸿沟的消除状况。
\end{itemize}}

\section{社会实践}
\cventry{18.12-\textbf{至今}}{SJTU中银俱乐部}{文宣部部长}{}{}{\begin{itemize}
   \item[•] 成立并带领部门对俱乐部日常活动进行跟踪记录,建立并维护俱乐部面向成员及公众的形象。
   \item[-] 和10名部门成员共同分担责任,安排工作,撰写新闻稿,拍摄照片与宣传视频,对外告知活动内容。
   \item[-] 运营社团官方微信公众号,推送关于金融与理财的系列科普知识。
\end{itemize}}

\cventry{17.12-19.04}{上海交通大学学生联合会跨文化交流中心}{秘书部部长}{}{}{\begin{itemize}
   \item[•] 负责各部门统筹协调、经费管理、人员流动,负责对外联系合作。
   \item[-] 策划跟进中心每年定期的罗德学者论坛,促进大学生国际学术交流研讨。
   \item[-] 代表学联中心与合作对象进行商务接洽;邀请外部专家学者进行友好交流。
\end{itemize}}

%\cvline{17.11}{\textbf{上海交大-国际传播学会(ICA)新媒体国际论坛}——嘉宾接待}
%\cvline{18.07}{\textbf{上海交大-国际传播学会(ICA)新媒体国际论坛}——论文集编辑校对工作}
%\cvline{18.05}{\textbf{上海科技节-科普大讲坛}——主持人}
%\cvline{12\textbf{年}-\textbf{至今}}{\textbf{微信个人公众号运营}}




%\cvline{markvis}{在 markdown 中直接生成可视化图表的插件
%  \emph{https://markvis.js.org} \textbf{GitHub 1000 stars}}
%\cvline{netjsongraph.js}{用力导向图可视化出无线路由图谱数据 \emph{https://github.com/netjson/netjsongraph.js}}
%\cvline{typing}{Hexo 静态博客主题 \emph{https://github.com/geekplux/hexo-theme-typing}}
%\cvline{UnityVis}{Unity 中的基本可视化图表 \emph{https://github.com/geekplux/Basic-Visualization-in-Unity}}

%\section{Publications}
%\cvline{已录用}{张三,李四,王麻子 基于 latex 的简历凑字数研究[C]// CVChina. 2017.}

% \subsection{Vocational}
% \cventry{year--year}{Job title}{Employer}{City}{}{Description}                % arguments 3 to 6 are optional
% \cventry{year--year}{Job title}{Employer}{City}{}{Description}                % arguments 3 to 6 are optional
% \subsection{Miscellaneous}
% \cventry{year--year}{Job title}{Employer}{City}{}{Description line 1\newline{}Description line 2}% arguments 3 to 6 are optional

% \section{Languages}
% \cvlanguage{language 1}{Skill level}{Comment}
% \cvlanguage{language 2}{Skill level}{Comment}
% \cvlanguage{language 3}{Skill level}{Comment}

% \section{Computer skills}
% \cvcomputer{category 1}{XXX, YYY, ZZZ}{category 4}{XXX, YYY, ZZZ}
% \cvcomputer{category 2}{XXX, YYY, ZZZ}{category 5}{XXX, YYY, ZZZ}
% \cvcomputer{category 3}{XXX, YYY, ZZZ}{category 6}{XXX, YYY, ZZZ}

% \section{Interests}
% \cvline{篮球}{\small 体力与技巧}
% \cvline{hobby 2}{\small Description}
% \cvline{hobby 3}{\small Description}

% \renewcommand{\listitemsymbol}{-} % change the symbol for lists

% \section{Extra 1}
% \cvlistitem{Item 1}
% \cvlistitem{Item 2}
%\cvlistitem[+]{Item 3}            % optional other symbol% XeLaTeX can use any Mac OS X font. See the setromanfont command below.
% Input to XeLaTeX is full Unicode, so Unicode characters can be typed directly into the source.

% The next lines tell TeXShop to typeset with xelatex, and to open and save the source with Unicode encoding.

%!TEX TS-program = xelatex
%!TEX encoding = UTF-8 Unicode

%\section{Extra 2}
%\cvlistdoubleitem[\Neutral]{Item 1}{Item 4}
%\cvlistdoubleitem[\Neutral]{Item 2}{Item 5}
%\cvlistdoubleitem[\Neutral]{Item 3}{}

%% Publications from a BibTeX file
%\nocite{*}
%\bibliographystyle{plain}
%\bibliography{publications}       % 'publications' is the name of a BibTeX file

% \begin{thebibliography}{}
% \bibitem[]{} 移动增强现实可视化综述[C]. ChinaVis 2017.
% \end{thebibliography}


\end{document}


%% end of file `template_en.tex'.

%%% Local Variables:
%%% mode: latex
%%% TeX-command-extra-options: "-shell-escape"
%%% TeX-master: t
%%% TeX-engine: xetex
%%% End:
